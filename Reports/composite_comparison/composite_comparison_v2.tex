% A summary of my results "so far" for the DART people in Boulder, March 9, 2012.
\documentclass[11pt]{article}
\usepackage[a4paper,margin=0.4in]{geometry}                % See geometry.pdf to learn the layout options. There are lots.
\usepackage[parfill]{parskip}    % Activate to begin paragraphs with an empty line rather than an indent
\usepackage{graphicx}
\usepackage{amssymb}
\usepackage{amsmath}
\usepackage{epstopdf}
\usepackage{natbib}
\DeclareGraphicsRule{.tif}{png}{.png}{`convert #1 `dirname #1`/`basename #1 .tif`.png}
\usepackage{pdflscape}

\title{Comparison of Composite ERPs and AAM Excitation Functions}

\date{\today}   
%% ------------------------------------------------------------------------ %%
%
%  MATH ABBREVIATSIONS
%
%% ------------------------------------------------------------------------ %%
\newcommand{\dlod}{{\Delta \text{LOD}}}
\newcommand{\peq}{{p}_{\text{eq}}}
\newcommand{\chieq}{{\chi}_{\text{eq}}}
\newcommand{\xm}{\chi^{\text{M}}}
\newcommand{\xw}{\chi^{\text{W}}}
\newcommand{\degN}{^{o}\text{N}}
\newcommand{\degS}{^{o}\text{S}}
\newcommand{\kgmsq}{\text{kgm}^{2}}
\newcommand{\dlodw}{\Delta \text{LOD}^{\text{W}}}
\newcommand{\dlodm}{\Delta \text{LOD}^{\text{M}}}

\begin{document}
\maketitle   

\bibliographystyle{plainnat}

% (0) Observations
\begin{figure}
  \noindent
  \includegraphics[width=\textwidth]{/home/ig/neef/Documents/Plots/SSW_AAM/composite_X1_obs_m_compare_subsets.png}
  \caption{Composites of observed PM1 (black) and the mass excitation function $\xm_1$ (yellow).  Each panel shows composites over different subsets of SSW events. For each curve, the solid line represents the mean and the shading the 98 $\%$ bootstrap confidence interval.}
   \label{fig:all_events}
 \end{figure}

\begin{figure}
  \noindent
  \includegraphics[width=\textwidth]{/home/ig/neef/Documents/Plots/SSW_AAM/composite_X1_obs_w_compare_subsets.png}
  \caption{Composites of observed PM1 (black) and the wind excitation function $\xw_1$ (blue).  Each panel shows composites over different subsets of SSW events. For each curve, the solid line represents the mean and the shading the 98 $\%$ bootstrap confidence interval.}
   \label{fig:all_events}
 \end{figure}

\begin{figure}
  \noindent
  \includegraphics[width=\textwidth]{/home/ig/neef/Documents/Plots/SSW_AAM/composite_X2_obs_m_compare_subsets.png}
  \caption{Composites of observed PM2 (black) and the mass excitation function $\xm_2$ (yellow).  Each panel shows composites over different subsets of SSW events. For each curve, the solid line represents the mean and the shading the 98 $\%$ bootstrap confidence interval.}
   \label{fig:all_events}
 \end{figure}

\begin{figure}
  \noindent
  \includegraphics[width=\textwidth]{/home/ig/neef/Documents/Plots/SSW_AAM/composite_X2_obs_w_compare_subsets.png}
  \caption{Composites of observed PM2 (black) and the wind excitation function $\xw_2$ (blue).  Each panel shows composites over different subsets of SSW events. For each curve, the solid line represents the mean and the shading the 98 $\%$ bootstrap confidence interval.}
   \label{fig:all_events}
 \end{figure}

\begin{figure}
  \noindent
  \includegraphics[width=\textwidth]{/home/ig/neef/Documents/Plots/SSW_AAM/composite_X3_obs_w_compare_subsets.png}
  \caption{Composites of observed $\dlod$ (black) and the wind excitation function $\xm_1$ (blue).  Each panel shows composites over different subsets of SSW events. For each curve, the solid line represents the mean and the shading the 98 $\%$ bootstrap confidence interval.}
   \label{fig:all_events}
 \end{figure}



\end{document}  
