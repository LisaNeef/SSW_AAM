% A summary of my results "so far" for the DART people in Boulder, March 9, 2012.
\documentclass[11pt]{article}
\usepackage[a4paper,margin=0.4in]{geometry}                % See geometry.pdf to learn the layout options. There are lots.
\usepackage[parfill]{parskip}    % Activate to begin paragraphs with an empty line rather than an indent
\usepackage{graphicx}
\usepackage{amssymb}
\usepackage{amsmath}
\usepackage{epstopdf}
\usepackage{natbib}
\DeclareGraphicsRule{.tif}{png}{.png}{`convert #1 `dirname #1`/`basename #1 .tif`.png}
\usepackage{pdflscape}

\title{Comparison of Composite AAM Excitation Functions}

\date{\today}   
%% ------------------------------------------------------------------------ %%
%
%  MATH ABBREVIATSIONS
%
%% ------------------------------------------------------------------------ %%
\newcommand{\dlod}{{\Delta \text{LOD}}}
\newcommand{\peq}{{p}_{\text{eq}}}
\newcommand{\chieq}{{\chi}_{\text{eq}}}
\newcommand{\xm}{\chi^{\text{M}}}
\newcommand{\xw}{\chi^{\text{W}}}
\newcommand{\degN}{^{o}\text{N}}
\newcommand{\degS}{^{o}\text{S}}
\newcommand{\kgmsq}{\text{kgm}^{2}}
\newcommand{\dlodw}{\Delta \text{LOD}^{\text{W}}}
\newcommand{\dlodm}{\Delta \text{LOD}^{\text{M}}}

\begin{document}
\maketitle   

\noindent \textbf{Some notes on the following plots:}

\begin{enumerate}
\item Selection of downward-propagating events from \citet{nakagawayamazaki2006}:  Events where the mean 30-day (after the central date) anomaly of 500hPa polar temperature is positive are called ``troposphere warm'' events (there are 14).  The rest are called ``troposphere cold'' events (there are 8).
%
\item Selection of downward-propagating events by Sophia: Events where the wind anomaly at 50hPa and 60N exceeds -15 m/s are called "strong events" (there are 10).  Events where the wind anomaly exceeds -20 m/s are called "stronger events" (there are 6).  
\end{enumerate}


\bibliographystyle{plainnat}



% (0) All SSW Events
\begin{figure}
  \noindent
  \includegraphics[width=\textwidth]{/home/ig/neef/Documents/Plots/SSW_AAM/composite_X1_all_events.png}
  \includegraphics[width=\textwidth]{/home/ig/neef/Documents/Plots/SSW_AAM/composite_X2_all_events.png}
  \includegraphics[width=\textwidth]{/home/ig/neef/Documents/Plots/SSW_AAM/composite_X3_all_events.png}
  \caption{Composites of the six AEFs, over all SSW events.}
   \label{fig:all_events}
 \end{figure}

% (1) Vortex-displacement events only
\begin{figure}
  \noindent
  \includegraphics[width=\textwidth]{/home/ig/neef/Documents/Plots/SSW_AAM/composite_X1_displ_events.png}
  \includegraphics[width=\textwidth]{/home/ig/neef/Documents/Plots/SSW_AAM/composite_X2_displ_events.png}
  \includegraphics[width=\textwidth]{/home/ig/neef/Documents/Plots/SSW_AAM/composite_X3_displ_events.png}
  \caption{Composites of the six AEFs, over the 12 vortex-displacement events.}
   \label{fig:displ_events}
 \end{figure}

% (2) Vortex-split events only
\begin{figure}
  \noindent
  \includegraphics[width=\textwidth]{/home/ig/neef/Documents/Plots/SSW_AAM/composite_X1_split_events.png}
  \includegraphics[width=\textwidth]{/home/ig/neef/Documents/Plots/SSW_AAM/composite_X2_split_events.png}
  \includegraphics[width=\textwidth]{/home/ig/neef/Documents/Plots/SSW_AAM/composite_X3_split_events.png}
  \caption{Composites of the six AEFs, over the 10 vortex-split events.}
   \label{fig:split_events}
 \end{figure}

% (3) Events where the u-anomaly at 50hPa is lower than -15 m/s
\begin{figure}
  \noindent
  \includegraphics[width=\textwidth]{/home/ig/neef/Documents/Plots/SSW_AAM/composite_X1_uanom_-15ms_events.png}
  \includegraphics[width=\textwidth]{/home/ig/neef/Documents/Plots/SSW_AAM/composite_X2_uanom_-15ms_events.png}
  \includegraphics[width=\textwidth]{/home/ig/neef/Documents/Plots/SSW_AAM/composite_X3_uanom_-15ms_events.png}
  \caption{Composites of the six AEFs, over the 10 events where the 50hPa (60N) u-anomaly is below -15 m/s.}
   \label{fig:uanom_events_1}
 \end{figure}

% (4) Events where the u-anomaly at 50hPa is lower than -20 m/s
\begin{figure}
  \noindent
  \includegraphics[width=\textwidth]{/home/ig/neef/Documents/Plots/SSW_AAM/composite_X1_uanom_-20ms_events.png}
  \includegraphics[width=\textwidth]{/home/ig/neef/Documents/Plots/SSW_AAM/composite_X2_uanom_-20ms_events.png}
  \includegraphics[width=\textwidth]{/home/ig/neef/Documents/Plots/SSW_AAM/composite_X3_uanom_-20ms_events.png}
  \caption{Composites of the six AEFs, over the 10 events where the 50hPa (60N) u-anomaly is below -20 m/s.}
   \label{fig:uanom_events_2}
 \end{figure}

% (5) Events classified as "troposphere warm" according to Nakagawa & Yamazaki (2006)
\begin{figure}
  \noindent
  \includegraphics[width=\textwidth]{/home/ig/neef/Documents/Plots/SSW_AAM/composite_X1_Nakagawa_strong_events.png}
  \includegraphics[width=\textwidth]{/home/ig/neef/Documents/Plots/SSW_AAM/composite_X2_Nakagawa_strong_events.png}
  \includegraphics[width=\textwidth]{/home/ig/neef/Documents/Plots/SSW_AAM/composite_X3_Nakagawa_strong_events.png}
  \caption{Composites of the six AEFs, over the 14 events classified as ``troposphere warm'' events according to the criterion of \citet{nakagawayamazaki2006}.}
   \label{fig:nakagawa_strong}
 \end{figure}

% (6) Events classified as "superstrong" according to Sophia, using the Nakagawa & Yamazaki (2006)
\begin{figure}
  \noindent
  \includegraphics[width=\textwidth]{/home/ig/neef/Documents/Plots/SSW_AAM/composite_X1_Nakagawa_weak_events.png}
  \includegraphics[width=\textwidth]{/home/ig/neef/Documents/Plots/SSW_AAM/composite_X2_Nakagawa_weak_events.png}
  \includegraphics[width=\textwidth]{/home/ig/neef/Documents/Plots/SSW_AAM/composite_X3_Nakagawa_weak_events.png}
  \caption{Composites of the six AEFs, over the 8 events classified as ``troposphere cold'' events according to the criterion of \citet{nakagawayamazaki2006}.}
   \label{fig:nakagawa_weak}
 \end{figure}

% (7) comparison of all composite subsets for $\chi_1$.
\begin{figure}
  \noindent
  \includegraphics[width=\textwidth]{/home/ig/neef/Documents/Plots/SSW_AAM/composite_X1_obs_compare_subsets.png}
  \caption{Composites of observed $p_1$, for each of the 7 SSW subsets.}
   \label{fig:X1_obs_subsets}
 \end{figure}

\begin{figure}
  \noindent
  \includegraphics[width=\textwidth]{/home/ig/neef/Documents/Plots/SSW_AAM/composite_X1m_compare_subsets.png}
  \caption{Composites of observed $p_1$ and the $\chi_1$ mass term, for each of the 7 SSW subsets.}
   \label{fig:X1_m_subsets}
 \end{figure}

\begin{figure}
  \noindent
  \includegraphics[width=\textwidth]{/home/ig/neef/Documents/Plots/SSW_AAM/composite_X1_obs_m_compare_subsets.png}
  \caption{Composites of the $\chi_1$ mass term, for each of the 7 SSW subsets.}
   \label{fig:X1_obs_m_subsets}
 \end{figure}


% (8) comparison of all composite subsets for $\chi_2$.
\begin{figure}
  \noindent
  \includegraphics[width=\textwidth]{/home/ig/neef/Documents/Plots/SSW_AAM/composite_X2_obs_compare_subsets.png}
  \caption{Composites of observed $p_2$, for each of the 7 SSW subsets.}
   \label{fig:X2_obs_subsets}
 \end{figure}

\begin{figure}
  \noindent
  \includegraphics[width=\textwidth]{/home/ig/neef/Documents/Plots/SSW_AAM/composite_X2m_compare_subsets.png}
  \caption{Composites of  the $\chi_2$ mass term, for each of the 7 SSW subsets.}
   \label{fig:X2_m_subsets}
 \end{figure}


\begin{figure}
  \noindent
  \includegraphics[width=\textwidth]{/home/ig/neef/Documents/Plots/SSW_AAM/composite_X2_obs_m_compare_subsets.png}
  \caption{Composites of observed $p_2$ and the $\chi_2$ wind term, for each of the 7 SSW subsets.}
   \label{fig:X2_obs_m_subsets}
 \end{figure}


% (9) comparison of all composite subsets for $\chi_3$.
\begin{figure}
  \noindent
  \includegraphics[width=\textwidth]{/home/ig/neef/Documents/Plots/SSW_AAM/composite_X3_obs_compare_subsets.png}
  \caption{Composites of observed $\dlod$, for each of the 7 SSW subsets.}
   \label{fig:X3_obs_subsets}
 \end{figure}

\begin{figure}
  \noindent
  \includegraphics[width=\textwidth]{/home/ig/neef/Documents/Plots/SSW_AAM/composite_X3w_compare_subsets.png}
  \caption{Composites of  the $\chi_3$ wind term, for each of the 7 SSW subsets.}
   \label{fig:X3_w_subsets}
 \end{figure}

\begin{figure}
  \noindent
  \includegraphics[width=\textwidth]{/home/ig/neef/Documents/Plots/SSW_AAM/composite_X3_obs_w_compare_subsets.png}
  \caption{Composites of observed $\dlod$ and the $\chi_3$ wind term, for each of the 7 SSW subsets.}
   \label{fig:X3_obs_w_subsets}
 \end{figure}

% (10) comparing the composite wind-anomalies over lat and time
\begin{landscape}

% (11) comparing wind anomalies in the different subsets
\begin{figure}
  \noindent
  \includegraphics[width=0.35\textwidth]{/home/ig/neef/Documents/Plots/Sophia/time_lev_wind60anom_tempdiffanom_compositmajor.png}
  \includegraphics[width=0.35\textwidth]{/home/ig/neef/Documents/Plots/Sophia/time_lev_wind60anom_tempdiffanom_compositsplit.png}
  \includegraphics[width=0.35\textwidth]{/home/ig/neef/Documents/Plots/Sophia/time_lev_wind60anom_tempdiffanom_composit15-stark.png}
  \includegraphics[width=0.35\textwidth]{/home/ig/neef/Documents/Plots/Sophia/time_lev_wind60anom_tempdiffanom_compositstrong-nakagawa.png} \\
  %
  \includegraphics[width=0.35\textwidth]{/home/ig/neef/Documents/Plots/Sophia/time_lev_wind60anom_tempdiffanom_compositmajor.png}
  \includegraphics[width=0.35\textwidth]{/home/ig/neef/Documents/Plots/Sophia/time_lev_wind60anom_tempdiffanom_compositdisplacement.png}
  \includegraphics[width=0.35\textwidth]{/home/ig/neef/Documents/Plots/Sophia/time_lev_windanom_tempdiffanom_composit15-schwach.png}
  \includegraphics[width=0.35\textwidth]{/home/ig/neef/Documents/Plots/Sophia/time_lev_wind60anom_tempdiffanom_compositweak-nakagawa.png} \\
  \caption{Composites of wind and and temperature anomalies over the different SSW subsets.}
   \label{fig:compare_windanom_composites}
 \end{figure}

% (12) comparing wind and temperature in the different subsets
\begin{figure}
  \noindent
  \includegraphics[width=0.35\textwidth]{/home/ig/neef/Documents/Plots/Sophia/time_lev_wind60_tempdiff_compositmajor.png}
  \includegraphics[width=0.35\textwidth]{/home/ig/neef/Documents/Plots/Sophia/time_lev_wind60_tempdiff_compositsplit.png}
  \includegraphics[width=0.35\textwidth]{/home/ig/neef/Documents/Plots/Sophia/time_lev_wind60_tempdiff_composit15-strong.png}
  \includegraphics[width=0.35\textwidth]{/home/ig/neef/Documents/Plots/Sophia/time_lev_wind60_tempdiff_compositstrong-nakagawa.png} \\
  %
  \includegraphics[width=0.35\textwidth]{/home/ig/neef/Documents/Plots/Sophia/time_lev_wind60_tempdiff_compositmajor.png}
  \includegraphics[width=0.35\textwidth]{/home/ig/neef/Documents/Plots/Sophia/time_lev_wind60_tempdiff_compositdisplacement.png}
  \includegraphics[width=0.35\textwidth]{/home/ig/neef/Documents/Plots/Sophia/time_lev_wind60_tempdiff_composit15-weak.png}
  \includegraphics[width=0.35\textwidth]{/home/ig/neef/Documents/Plots/Sophia/time_lev_wind60_tempdiff_compositweak-nakagawa.png} \\
  \caption{Composites of wind and and temperature over the different SSW subsets.}
  \label{fig:compare_wind_composites}
 \end{figure}

\end{landscape}

% (13) comparing regional contributions to the AEFs

\begin{figure}
  \noindent
  \includegraphics[width=\textwidth]{/home/ig/neef/Documents/Plots/SSW_AAM/composite_X1m_compare_regions.png}
  \includegraphics[width=\textwidth]{/home/ig/neef/Documents/Plots/SSW_AAM/composite_X2m_compare_regions.png}
  \includegraphics[width=\textwidth]{/home/ig/neef/Documents/Plots/SSW_AAM/composite_X3w_compare_regions.png}
  %
  \caption{Composites of the dominant terms of each AEF, comparing integrationover the tropics only (center), and the NHET only (right).}
 \label{fig:compare_regions}
\end{figure}




\bibliography{ssw}

\end{document}  
