\documentclass[a4paper,10pt]{article}
%\documentclass[draft,jgrga]{agutex}
\usepackage[utf8]{inputenc}

%opening
\title{Response to Reviewers  (Second Round)}
\author{}


% packages
\usepackage[dvips]{graphicx}
\usepackage{amsmath}
\usepackage[usenames]{color}
\usepackage{soul}
\usepackage{ulem}
\usepackage{url}
\usepackage{lscape}
\usepackage{epstopdf}
\usepackage{natbib}

% colors
\definecolor{misc2}{RGB}{27,158,119}
\definecolor{misc1}{RGB}{217,95,2}
\definecolor{reviewer}{RGB}{117,112,179}

\usepackage{geometry}
 \geometry{
 a4paper,
 total={210mm,297mm},
 left=20mm,
 right=20mm,
 top=20mm,
 bottom=20mm,
 }



\begin{document}
%\bibliographystyle{agu08}
\bibliographystyle{plainnat}


\maketitle


We would again like to thank all three reviewers for their careful reading of the manuscript and helpful comments and discussion.


%-------------------------------------------------------------------------
\section{Response to Comments by Reviewer 1}
\label{sec:R1}


\textcolor{reviewer}{The legend of Figure 3 notes 4 rows in the figure, but there are only 3. I think the third part of the legend should be removed, and the fourth part become a third part. }
\vspace{0.5cm}

This is correct; we have made the change.
\vspace{0.5cm}

%-------------------------------------------------------------------------
\section{Response to Comments by Reviewer 2}
\label{sec:R2}



\textcolor{reviewer}{50: "components of AAM the orientation"...is there a word missing here?}
\vspace{0.5cm}

We have added the word ``change'' to make the sentence more clear.  
\vspace{0.5cm}

\noindent \textcolor{reviewer}{Paragraph at 83: is not mentioning Section 3 intentional? Fine if so.}
\vspace{0.5cm}

We omitted this by accident; we've now corrected the text to make sure the outline mentions all sections.
\vspace{0.5cm}


\noindent \textcolor{reviewer}{306: "shows a such strong"}
\vspace{0.5cm}

We have made the correction.
\vspace{0.5cm}

\noindent \textcolor{reviewer}{348: "As both of surface pressure patterns contribute"}
\vspace{0.5cm}

We have made the correction.
\vspace{0.5cm}

\noindent \textcolor{reviewer}{375-376: this sentence is quite confusing, especially "westerly wind anomalies that might precede SSWs tropical troposphere"}
\vspace{0.5cm}

We have modified broken the sentence up into two sentences and modified the wording to make it more clear.
\vspace{0.5cm}

\noindent \textcolor{reviewer}{Figure 4: the caption appears to refer to 4 different rows of plots}
\vspace{0.5cm}

We have corrected the caption and removed the erroneous reference to a non-existant row of plots.
We have made the correction.
\vspace{0.5cm}



%-------------------------------------------------------------------------
\section{Response to Comments by Reviewer 3}
\label{sec:R3}

\textcolor{reviewer}{First, the author’s included several ‘review only’ figures and I am concerned about the
accuracy of either their data or their methods. Specifically, figures 2 and 4 have some
very peculiar characteristics. All of the lines plotted prior to and including 1981 are very
smooth, while the lines plotted beyond 1981 are very noisy. At first I was thinking that
this might have to do with the switching of the data sets, but the author’s state that ERA-
Interim is used beyond 1979, so this clearly can’t be the reason for the discrepancy.
Given the very different appearance of the data, I am concerned that the other results –
including Figure 2 – may be inaccurate. Can the author’s please clarify what is going on
here?}
\vspace{0.5cm}

The difference between the Earth rotation parameters curves  for pre-1981 and post-1981 events is due to the difference in precision of the ERP products --- starting around 1981, the ERPs were computed from a combination of ground- based and spaceborne geodesy, and therefore have a much higher precision and are able to resolve day-to-day variations in both polar motion and LOD.  We neglected to mention this in our first response, but since this is an important aspect of the data, we have added a sentence about the evolving accuracy of the ERPs to the end of Section 2.1. 
\vspace{0.5cm}


\textcolor{reviewer}{Second, given my comments above about the timing and strength of the PM2 signals
shown for split versus displacement SSWs (see the end of my General Comments section above),
I think it would be appropriate for the author’s to briefly note in the text that there
are some differences between split and displacement warmings and that their results do
seem to depend on the criteria used for picking warming events (i.e. the Charlton/Polvani
versus Mitchell et al. criteria). Its fine if the author’s want to point out in their manuscript
that they have not explored this avenue because of the small sample size, but I think that
they need to make readers aware that there are indeed potential differences between splits
and displacements. Perhaps the author’s can add something to their text either near lines
383 or 275.}
\vspace{0.5cm}


We have expanded the discussion at the end of Section 4.2 to highlight the differences found between splitting and displacement events in the literature, and we now mention that a difference between the types of events is visible (but not statistically significant) in the ERP data.
We have also brought in a reference to the recent study  by Barriopedro and Calvo (2014), which highlights the complexity of the precursor patterns associated with each event type.  

In the conclusion section, we have added an acknowledgement that a closer look at the difference between splitting and displacement events in terms of AAM would be worth taking in the future.  

\end{document}

